\documentclass[11pt,a4paper]{article}
\usepackage[margin=1in]{geometry}
\usepackage{graphicx}
\usepackage{booktabs}
\usepackage{amsmath}
\usepackage{hyperref}
\usepackage[usenames,dvipsnames]{xcolor}
\usepackage{float}
\usepackage{caption}
\usepackage{subcaption}
\usepackage[style=apa,backend=biber]{biblatex}
\usepackage{setspace}

\hypersetup{colorlinks=true, linkcolor=NavyBlue, citecolor=NavyBlue, urlcolor=NavyBlue}
\onehalfspacing

\title{%
  Portable EEG Correlates of Basketball Free-Throw Performance:\\
  Pilot Session Analysis Report
}
\author{Lukas Laskowski \and Kyle E.\ Mathewson}
\date{%%DATE%%}

\begin{document}
\maketitle

% ────────────────────────────────────────────────────────────────────────────
\begin{abstract}
This report presents the analysis of a single pilot session recorded using the
FreethrowEEG system. Continuous EEG band power (delta, theta, alpha, beta,
gamma) was recorded from a Muse~2 headset while the participant performed
%%NTOTAL%% free throws (%%NMADE%% made, %%NMISSED%% missed; %%SHOOTPCT%%\%).
Shot-locked epochs were extracted and compared between successful and
unsuccessful attempts. Concurrent video was analysed with MediaPipe pose
estimation to extract biomechanical features aligned with neural dynamics.
Descriptive statistics, time-series visualizations, pose kinematics, and
exploratory inferential tests are reported. %%ABSTRACT_EXTRA%%
\end{abstract}

% ────────────────────────────────────────────────────────────────────────────
\section{Introduction}

Electroencephalography (EEG) has been widely used to study neural correlates of
motor performance, including precision sport tasks such as archery, shooting,
and golf putting. A consistent finding in the literature is that successful
performers exhibit distinct pre-performance EEG patterns, particularly reduced
alpha desynchronization and modulated theta activity, compared to unsuccessful
attempts. This pilot session tests whether such patterns are detectable using a
portable, four-channel Muse~2 headset during basketball free-throw shooting.

The present analysis focuses on:
\begin{enumerate}
  \item Visualizing continuous and shot-locked EEG band power;
  \item Comparing pre-shot, execution, and post-shot neural signatures between
        made and missed free throws;
  \item Evaluating the theta/alpha ratio as a candidate readiness marker;
  \item Examining shot-by-shot band power progression across the session.
\end{enumerate}

% ────────────────────────────────────────────────────────────────────────────
\section{Methods}

\subsection{Participant and Task}
The participant (%%PLAYER%%) performed %%NTOTAL%% free throws during a single
session lasting %%DURATION%% minutes. Each trial followed a structured protocol:
a preparation phase ($\sim$5\,s), a pre-shot baseline ($\sim$5\,s), shot
execution ($\sim$2\,s), post-shot recording ($\sim$3\,s), and a review phase.
Shot outcome (made/missed) was manually coded.

\subsection{EEG Recording}
EEG was recorded continuously using a Muse~2 headband (four channels: TP9,
AF7, AF8, TP10; 256\,Hz sampling rate). Band power was computed in real time
using a 1-second sliding window with 25\% overlap. A 4th-order Butterworth
bandpass filter was applied to extract five standard frequency bands: delta
(1--4\,Hz), theta (4--8\,Hz), alpha (8--13\,Hz), beta (13--30\,Hz), and gamma
(30--50\,Hz). Environmental line noise was removed via a Butterworth notch
filter at 60\,Hz. Power values were averaged across all four channels.

\subsection{Video Recording and Pose Estimation}
Video was recorded simultaneously at 30\,fps (640$\times$480, WebM/VP9 codec)
using the device camera. The recording started with the session and timestamps
in the EEG data file map directly to video time. Post-hoc pose estimation was
performed using the MediaPipe Pose Landmarker (heavy model), extracting 33
body landmarks per frame. Biomechanical features---elbow angle
(shoulder--elbow--wrist), wrist height, knee angle (hip--knee--ankle), and body
lean angle (torso deviation from vertical)---were computed for each frame
during the shot execution phase. Ball release was estimated as the frame where
wrist velocity peaked while the wrist was near its highest position.

\subsection{Analysis}
Continuous EEG data were segmented into shot-locked epochs aligned to the onset
of the execution phase. Epochs were interpolated onto a common time grid
($-$12 to $+$8\,s relative to shot onset at 0.25\,s resolution) and averaged
separately for made and missed shots. Phase-wise mean power was compared using
Welch's $t$-test (unequal variances) and the Mann--Whitney $U$ test. Effect
sizes are reported as Cohen's $d$. Given the pilot nature of this data set
(%%NTOTAL%% shots), all inferential statistics should be interpreted with
caution.

Individual shot clips were extracted from the video, and stop-motion frame
sequences were generated for each trial. Average frame composites were created
by pixel-averaging release-point frames within each outcome group. ``Ghost''
overlays show the variability in shooting form by superimposing skeleton
renderings from all shots within a group.

% ────────────────────────────────────────────────────────────────────────────
\section{Results}

\subsection{Session Overview}
The participant completed %%NTOTAL%% shots over %%DURATION%% minutes
(%%SHOOTPCT%%\% accuracy). Figure~\ref{fig:raster} shows the session timeline
with shot phases and outcomes overlaid on the continuous alpha power trace.

\begin{figure}[H]
  \centering
  \includegraphics[width=\textwidth]{figures/fig_shot_raster.pdf}
  \caption{Session timeline. Each vertical line marks a shot (green = made,
  red = missed). Shaded regions indicate trial phases.}
  \label{fig:raster}
\end{figure}

\subsection{Continuous Band Power}
Figure~\ref{fig:continuous} shows the five frequency bands across the full
session. %%CONTINUOUS_INTERP%%

\begin{figure}[H]
  \centering
  \includegraphics[width=\textwidth]{figures/fig_continuous_power.pdf}
  \caption{Continuous band power (delta through gamma) over the entire session.
  Dashed vertical lines indicate shot execution (green = made, red = missed).}
  \label{fig:continuous}
\end{figure}

\subsection{Signal Filtering}
Figure~\ref{fig:filtering} demonstrates the effect of low-pass, high-pass, and
bandpass Butterworth filtering applied to the raw delta-band signal from the
first 60 seconds of the session.

\begin{figure}[H]
  \centering
  \includegraphics[width=\textwidth]{figures/fig_raw_filtering.pdf}
  \caption{Filtering demonstration on the raw delta signal: (a) unfiltered,
  (b) low-pass $<$1\,Hz, (c) high-pass $>$0.5\,Hz,
  (d) bandpass 0.5--1.5\,Hz.}
  \label{fig:filtering}
\end{figure}

\subsection{Shot-Locked Averages}
Figure~\ref{fig:shotlocked} shows mean band power ($\pm$ SEM) locked to the
onset of the execution phase, averaged across all %%NTOTAL%% shots.

\begin{figure}[H]
  \centering
  \includegraphics[width=\textwidth]{figures/fig_shot_locked_average.pdf}
  \caption{Shot-locked band power averages (all shots). Time 0 = execution
  onset. Shaded regions represent $\pm$1 SEM.}
  \label{fig:shotlocked}
\end{figure}

\subsection{Made vs.\ Missed Comparison}
Figure~\ref{fig:madevsmissed} compares shot-locked averages for made and missed
shots. %%MADEVSMISSED_INTERP%%

\begin{figure}[H]
  \centering
  \includegraphics[width=\textwidth]{figures/fig_success_vs_failure.pdf}
  \caption{Shot-locked power comparison: made (green) vs.\ missed (red).
  Shaded regions are $\pm$1 SEM.}
  \label{fig:madevsmissed}
\end{figure}

\subsection{Phase-Wise Analysis}
Figure~\ref{fig:phasebars} displays mean power during the pre-shot, execution,
and post-shot phases, broken down by outcome and frequency band.
%%PHASEBARS_INTERP%%

\begin{figure}[H]
  \centering
  \includegraphics[width=\textwidth]{figures/fig_phase_bars.pdf}
  \caption{Mean band power by phase and outcome ($\pm$ SEM).}
  \label{fig:phasebars}
\end{figure}

\subsection{Statistical Comparisons}
Table~\ref{tab:stats} summarizes pre-shot band power for made and missed shots,
along with inferential test results.

\begin{table}[H]
\centering
\caption{Pre-shot band power: descriptive and inferential statistics.}
\label{tab:stats}
\small
\begin{tabular}{lcccccc}
\toprule
Band & Made $M$ (SD) & Missed $M$ (SD) & $t$ & $p$ & $U$ & $d$ \\
\midrule
%%STATS_TABLE_ROWS%%
\bottomrule
\end{tabular}
\end{table}

\subsection{Theta/Alpha Ratio}
%%THETA_ALPHA_INTERP%%

\begin{figure}[H]
  \centering
  \includegraphics[width=\textwidth]{figures/fig_theta_alpha_ratio.pdf}
  \caption{(Top) Continuous $\theta/\alpha$ ratio over the session. (Bottom)
  Pre-shot $\theta/\alpha$ ratio by outcome.}
  \label{fig:thetaalpha}
\end{figure}

\subsection{Shot-by-Shot Progression}
Figure~\ref{fig:progression} tracks pre-shot band power across the 10 shots.
%%PROGRESSION_INTERP%%

\begin{figure}[H]
  \centering
  \includegraphics[width=\textwidth]{figures/fig_shot_progression.pdf}
  \caption{Pre-shot band power across shots. Dot colour indicates outcome
  (green = made, red = missed). Dashed line shows linear trend.}
  \label{fig:progression}
\end{figure}

% ────────────────────────────────────────────────────────────────────────────
\section{Video and Pose Estimation Analysis}

In addition to EEG, the session video was recorded continuously at 30\,fps
(640$\times$480 resolution). Post-hoc pose estimation was performed using
MediaPipe Pose (heavy model) to extract 33 body landmarks per frame.
Biomechanical features---elbow angle, wrist height, knee angle, and body lean
angle---were computed across the shot execution phase for each trial and
compared between made and missed shots.

\subsection{Shot Stop-Motion Montage}
Figure~\ref{fig:montage} presents a frame-by-frame stop-motion decomposition
of each shot during the execution phase, with pose skeleton overlays. Made
shots are shown in the upper rows (green borders) and missed shots in the
lower rows (red borders).

\begin{figure}[H]
  \centering
  \includegraphics[width=\textwidth]{figures/fig_shot_montage.pdf}
  \caption{Stop-motion montage of all shots during the execution phase. Pose
  skeleton overlays are drawn semi-transparently. Border colour indicates
  outcome (green = made, red = missed).}
  \label{fig:montage}
\end{figure}

\subsection{Ghost Shot Overlay}
Figure~\ref{fig:ghost} shows composite ``ghost'' images created by pixel-averaging
the release-point frames across made and missed shots separately (top row),
along with skeleton-only overlays showing individual shot pose variability
(bottom row).

\begin{figure}[H]
  \centering
  \includegraphics[width=\textwidth]{figures/fig_ghost_overlay.pdf}
  \caption{Ghost overlay composites at the estimated release point. Top:
  pixel-averaged frames for made (left) and missed (right) shots. Bottom:
  skeleton-only overlays with individual shots in distinct colours.}
  \label{fig:ghost}
\end{figure}

\subsection{Pose Kinematics Trajectories}
Figure~\ref{fig:kinematics} plots the time course of four biomechanical
features during the execution phase, comparing made (green) and missed (red)
shots. %%KINEMATICS_INTERP%%

\begin{figure}[H]
  \centering
  \includegraphics[width=\textwidth]{figures/fig_pose_kinematics.pdf}
  \caption{Biomechanical feature trajectories during shot execution (mean
  $\pm$ SEM). Thin lines show individual shots; thick lines show group means.
  Dashed vertical line: estimated release point.}
  \label{fig:kinematics}
\end{figure}

\subsection{Integrated Pose and EEG Analysis}
Figure~\ref{fig:poseeeg} presents a joint visualization of pose kinematics
(elbow angle, wrist height) and EEG dynamics (alpha power, $\theta/\alpha$
ratio) during the execution phase, time-aligned to recording onset.
%%POSEEEG_INTERP%%

\begin{figure}[H]
  \centering
  \includegraphics[width=\textwidth]{figures/fig_pose_eeg_combined.pdf}
  \caption{Combined pose kinematics and EEG dynamics during shot execution.
  Top row: elbow angle and wrist height. Bottom row: alpha band power and
  $\theta/\alpha$ ratio. All traces show made (green) vs.\ missed (red) mean
  $\pm$ SEM.}
  \label{fig:poseeeg}
\end{figure}

\subsection{Average Frame Composites}
Figure~\ref{fig:avgframes} shows pixel-averaged video frames at three time
points (start, midpoint, end) of the execution phase for made and missed shots,
with an absolute difference image highlighting regions of greatest divergence.

\begin{figure}[H]
  \centering
  \includegraphics[width=\textwidth]{figures/fig_average_frames.pdf}
  \caption{Average frame composites. Top row: made shots. Middle row: missed
  shots. Bottom row: absolute pixel difference. Columns correspond to
  execution start, midpoint, and end.}
  \label{fig:avgframes}
\end{figure}

% ────────────────────────────────────────────────────────────────────────────
\section{Discussion}

%%DISCUSSION%%

\subsection{Limitations}
This analysis is based on a single pilot session with %%NTOTAL%% shots
(%%NMADE%% made), which severely limits statistical power. The Muse~2 provides
only four channels, precluding source localization or detailed topographic
analysis. Band power values represent averages across all channels, which may
obscure hemisphere-specific effects. Future sessions with larger sample sizes
are needed to draw reliable conclusions.

% ────────────────────────────────────────────────────────────────────────────
\section{Conclusion}

This pilot analysis demonstrates that the FreethrowEEG system can successfully
record and analyse EEG band power during free-throw shooting with a portable
Muse headset. %%CONCLUSION_EXTRA%% The addition of video-based pose estimation
provides a complementary kinematic dimension, enabling time-aligned analysis of
brain and body dynamics during shot execution. Ghost overlays and average frame
composites offer intuitive visual summaries of shooting form variability. These
results provide a foundation for larger-scale data collection and more rigorous
hypothesis testing.

\end{document}
